\section{Results}
\label{sec:results}
This section details the results of the experiment, which was set up as described by the previous section.
The purpose of this section is to state how each tool performed and discussion of the results is deferred to the next section.
The reported metrics are the analysis time and the number of blocked system calls.

\subsection {Dynamic analysis solution}
As described in the experiment setup, the dynamic tracer supports both single phase and multi phase model of execution.
The following subsections describe the results for each of these 2 models.

\subsubsection {Single phase model}
\begin{table}[!h]
\begin{center}
\caption{Blocked system call count and analysis runtime for the dynamic analysis solution implemented for this research, using the single phase execution model}
\label{tbl:dyn_results}
\begin{tabular}{||c c c||} 
 \hline
 Program & Blocked system call count & Runtime \\
 \hline\hline
 ls & 345 & 104ms \\ 
 \hline
 sqlite & 341 & 344ms \\ 
 \hline
 redis & 346 & 652ms \\
 \hline
\end{tabular}
\end{center}
\end{table}

The dynamic analysis based tracer, written for this research is quick, however it blocks many system calls.
The analysis for \textit{ls} took around 100ms for executing 8 cases and gathering all results.
Out of the 373 system calls available on the host system it blocks 345, as depicted in Table \ref{tbl:dyn_results}, blocking system calls frequently used in attacks such as \textit{mprotect} and \textit{execve}.

For sqlite and redis the performance is similar, in the 300-400 ms range, blocking 341 and 346 system calls respectively. Just like in the case for \textit{ls}, system calls frequently used when exploiting binaries have been blocked, however due to the nature of the applications the \textit{read} and \textit{write} system calls still allow for some data oriented attacks.

\subsubsection {Multi phase model}
\begin{table}[!h]
\begin{center}
\caption{Blocked system call count and analysis runtime for the dynamic analysis solution implemented for this research, using the multi phase execution model. The second column depcits the number of blocked systems calls in the first phase, followed by the second phase.}
\label{tbl:dyn_mp_results}
\begin{tabular}{||c c c||}
 \hline
Program & Blocked syscalls per phase & Runtime \\
 \hline\hline
 ls & 345 / 354 & 108ms \\
 \hline
 sqlite & 341 / 357 & 361ms \\
 \hline
 redis & 346 / 357 & 725ms \\
 \hline
\end{tabular}
\end{center}
\end{table}
It is important to note, that all system calls made by phase 1 are also included in the allowed system calls of phase 0. This means that although phase 0 only uses 17 unique system calls, it also consideres system calls that are used exclusively by phase 1, hence the lower number of blocked system calls.
This is done because of how \textit{seccomp} operates, which is further detailed in the discussion.

In the case of \textit{ls}, considering a second execution phase allows for blocking 9 more system calls after phse 0 is done, as depicted in \ref{tbl:dyn_mp_results}, and finishes within 108ms, similar to the speed of the single phase model.

For sqlite, the increase in blocked system calls is even higher, with 16 additional calls blocked in the second phase and analysis finishing within 361ms.
In the case of redis 11 additional system calls are blocked and anlysis is done in 725ms.

\subsection {sysfilter}
\begin{table}[!h]
\begin{center}
\caption{Blocked system call count and analysis runtime for sysfilter.}
\label{tbl:sysfilter_results}
\begin{tabular}{||c c c||} 
 \hline
 Program & Blocked system call count & Runtime \\
 \hline\hline
 ls & 293 & 12.4s \\ 
 \hline
 sqlite & 269 & 17.35s \\ 
 \hline
 redis & 275 & 13.06s \\ 
 \hline
\end{tabular}
\end{center}
\end{table}
The analysis for \textit{ls} took around 12 seconds, identifying 293 system calls to be blocked, as depicted in Table \ref{tbl:sysfilter_results}.
Analysis of sqlite took slightly more time with 17 seconds and managed to block 269 system calls.
For redis the analysis finished a bit faster with 13 seconds, blocking 275 system calls.

\subsection {Confine}
\begin{table}[!h]
\begin{center}
\caption{Blocked system call count and analysis runtime for Confine.}
\label{tbl:confine_results}
\begin{tabular}{||c c c||} 
 \hline
 Program & Blocked system call count & Runtime \\
 \hline\hline
 ls & 201 & 1.38s \\ 
 \hline
 sqlite & 188 & 2.21s \\ 
 \hline
 redis & 189 & 1.54s \\ 
 \hline
\end{tabular}
\end{center}
\end{table}
\textit{ls} was analysed in around 1.3 seconds, identifying 201 system calls to be blocked, as shown in Table \ref{tbl:confine_results}.
Analysis for sqlite took slightly longer, with 2.2 seconds, resulting in 188 system calls to be blocked.
With redis the analysis took a bit less time with 1.5 seconds, and 189 system calls were blocked.

\subsection {Chestnut}
The following section details the results for the Chestnut project.
Since it has multiple components and the usage of statically linked binaries compared to dynamically linked binaries makes a significant difference in some results, more subsections are suitable for this project.

\subsubsection {Sourcealyzer}
\begin{table}[!h]
\begin{center}
\caption{Blocked system call count and analysis runtime for chestnut (Sourcealyzer).}
\label{tbl:chestnut_src_results}
\begin{tabular}{||c c c||} 
 \hline
 Program & Blocked system call count & Runtime \\
 \hline\hline
 ls & 318 & 4.53s \\ 
 \hline
 sqlite & 296 & 1.67s \\ 
 \hline
 redis & 286 & 4.54s \\ 
 \hline
\end{tabular}
\end{center}
\end{table}
Sourcealyzer produces 318 blocked system calls in around 4.5 seconds, when compiling \textit{ls}, as Table \ref{tbl:chestnut_src_results}.
For sqlite the analysis time drops to around 1.7 seconds, and 296 system calls are blocked as a result.
Redis took around 4.5 seconds to analyze, after which 286 system calls got blocked.

\subsubsection {Binalyzer - syscalls}
\begin{table}[!h]
\begin{center}
\caption{Blocked system call count and analysis runtime for chestnut (Binalyzer - syscalls).}
\label{tbl:chestnut_bin_sys_results}
\begin{tabular}{||c c c||} 
 \hline
 Program & Blocked system call count & Runtime \\
 \hline\hline
 ls (dynamic) & 93 & 6.37s \\ 
 \hline
 ls (static) & 319 & 951ms \\ 
 \hline
 sqlite (dynamic) & 99 & 10.768s \\ 
 \hline
 sqlite (static) & 298 & 3.01s \\ 
 \hline
 redis (dynamic) & 99 & 9.49s \\ 
 \hline
 redis (static) & 292 & 3.05s \\ 
 \hline
\end{tabular}
\end{center}
\end{table}
In this subsection binaries were analyzed using the \textit{syscalls.py} script.
For the dynamically linked \textit{ls} the analysis took 6.7 seconds and blocked only 93 system calls, in comparison the same analysis on the statically linked binary took 900 ms, but managed to block 319 system calls, as Table \ref{tbl:chestnut_bin_sys_results} depicts.

A similar trend between the static and dynamic linkage can be observed with the other 2 programs, with the dynamic sqlite binary taking 10 seconds to analyze and blocking 99 system calls and the static version taking only 3 seconds, and blocking 298 system calls.

Redis took 9 seconds to analyze using the dynamic version, but only 3 seconds when using the static version. 99 and 292 systems calls got blocked respectively.

\subsubsection {Binalyzer - cfg}
\begin{table}[!h]
\begin{center}
\caption{Blocked system call count and analysis runtime for chestnut (Binalyzer - cfg).}
\label{tbl:chestnut_bin_cfg_results}
\begin{tabular}{||c c c||} 
 \hline
 Program & Blocked system call count & Runtime \\
 \hline\hline
 ls (dynamic) & - & 5.52s \\ 
 \hline
 ls (static) & 318 & 11.88s \\ 
 \hline
 sqlite (dynamic) & - & 1.43s \\ 
 \hline
 sqlite (static) & 297 & 40.22s \\ 
 \hline
 redis (dynamic) & - & 53.3s \\ 
 \hline
 redis (static) & 292 & 60.52s \\ 
 \hline
\end{tabular}
\end{center}
\end{table}
In this subsection binaries were analyzed using the \textit{cfg.py} script.
For all 3 programs the analysis of the dynamically linked libraries did not yield any blocked system calls, with \textit{ls} taking 5 seconds to analyze and redis taking 53 seconds. Additionally for sqlite the analysis fails 1.4 seconds, due to \textit{angr} not being able to construct the CFG.

For the statically linked binaries the performance is better.
\textit{ls} took 11.8 seconds to analyze and 318 system calls got blocked, as shown in Table \ref{tbl:chestnut_bin_cfg_results}.
For sqlite the analysis took 40.2 seconds and blocked 297 system calls.
In the case of redis the analysis finished in 1 minute and blocked 292 system calls.

\subsection {Temporal specialization}
\begin{table}[!h]
\begin{center}
\caption{Blocked system call count and analysis runtime for temporal specialization. The second column depcits the number of blocked systems calls in the first phase, followed by the second phase. The thid column contains the runtime for the bitcode generation, followed by the runtime for the full analysis}.
\label{tbl:tmp_spec_results}
\begin{tabular}{||c c c||}
 \hline
 Program & Blocked syscalls per phase & Runtime \\
 \hline\hline
 sqlite & 305 / 311 & 1.91s / 2m 18.37s \\
 \hline
 redis & 287 / 295 & 5.97s / 2m 12.78s \\
 \hline
\end{tabular}
\end{center}
\end{table}
As explained in the experiment setup, \textit{ls} was not analyzed using this tool, as there was no good candidate function to mark as the worker phase.

In the case of sqlite, adding a second phase allows to block 6 additional system calls, as show in Table \ref{tbl:tmp_spec_results}. Bitcode generation finished under a reasonable 1.91 seconds, however the analysis part took over 2 minutes.

For redis, 8 additional system calls can be blocked in the second phase. Bitcode generation took around 6 seconds, while the analysis can last over 2 minutes, like in the case of sqlite.
